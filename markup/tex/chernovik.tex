% ОБЯЗАТЕЛЬНО ИМЕННО ТАКОЙ documentclass!
% (Основной кегль = 14pt, поэтому необходим extsizes)
% Формат, разумеется, А4
% article потому что стандарт не подразумевает разделов
% Глава = section, Параграф = subsection
% (понятия "глава" и "параграф" из стандарта)
\documentclass[a4paper,article,14pt]{extarticle}

% Подключаем главный пакет со всем необходимым
\usepackage{spbudiploma}

% Пакеты по желанию (самые распространенные)
% Хитрые мат. символы
\usepackage{euscript}
% Таблицы
\usepackage{longtable}
\usepackage{makecell}
% Картинки (можно встявлять даже pdf)
\usepackage[pdftex]{graphicx}

\usepackage{amsthm,amssymb, amsmath}
\usepackage{textcomp}

\begin{document}

% Титульник в файле titlepage.tex
\input{titlepage.tex}

% Содержание
\tableofcontents
\pagebreak

\specialsection{Введение}

	Термин Web 2.0 впервые был использован в статье \flqq Tim O’Reilly — What Is Web 2.0\frqq от 30 сентября 2005 года. Под этим термином подразумевалась общая тенденция развития интернет-сообщества. Суть этого термина отображает комплексный подход к организации, реализации и поддержке Web-ресурсов. Сейчас под сетями Web 2.0 подразумевают сети, которые путём учёта сетевых взаимодействий становятся тем полнее, чем больше людей ими пользуются. Особенностью Web 2.0 является принцип привлечения пользователей к наполнению.

	К сетям Web 2.0 впервую очередь относятся социальные сети, мессенджеры и форумы. Но подходы Web 2.0 распространены также и в сетях, не ставящих своей целью обмен пользовательским контентом, например, онлайн-магазины и новостные сервисы также используют подходы Web 2.0: возможность оставлять комментарии и писать свои посты, подписываться на интересующие аккаунты, заполнять профиль, писать сообщения, составлять собственную новостную ленту и т.д.

	Самое большое распространение получили социальные сети. Они различаются типом контента и способами взаимодействия пользователей. Но основным принципом остаётся - наполнение сети пользовательским контентом. Такой подход получил название UGC (User Generated Content). Пользователи помимо генерации контента могут выражать свои эмоции, оставляя реакции под постами, присоединяться к сообществам, подписываясь на каналы или вступая в чаты, выражать собственное мнение, используя упоминания, цитирование, репосты и комментарии.

	Все вышеперечисленные способы взаимодействия поддаются анализу. Пользователи в процессе взаимодействия друг с другом создают дискуссию, определяют её тональность, выражают проблематику, связывают с другими дискуссиями. В такой дискуссии уже можно выделить лидеров мнений, стороны обсуждения, основную боль пользователей, объект дискуссии и общую заинтересованность, или остроту проблемы. И всё это происходит неосознанно, а иногда и наоборот, осознанно, преследуя определённые цели и мотивы.

	Проанализировав такую дискуссию можно получить важную информацию о текущем положении дел или общее отношение пользователей к конкретной тематике, продукту, событию или идее. Более того, разнообразие подходов для взаимодействия даёт возможность составить географическую карту настроения пользователей. К таким подходам относятся гео-чаты (чаты привязанные к геопозиции), локальные чаты (чаты сообществ и муниципалитетов) и геометки конкретных постов.

	Получение списка пользователей сообществ даёт возможность найти пересечение интересов пользователей, скрытые сообщества и источники мнений, перетекающих из одного обсуждения в другое. Анализ активности и взаимодействия с другими пользователями может указать на транслирование определённой информации преследуя личные мотивы, распространение пропаганды или \flqq вбросов\frqq - заведомо ложной информации, в целях создания общей паники, распространению продукта или в иных целях.

	Граф связей сообщений позволяет выделить объект дискуссии, определить взаимосвязанные дискуссии, из которых, или в которые, перетекло обсуждение. Иногда боль пользователя или источник его утверждений можно определить лишь выделив частную дискуссию или идущую параллельно, т.е. дискуссию по другой тематике, имеющей косвенную связь с текущей.

\specialsection{Актуальность}

	Рынок сетей Web 2.0 уже занимает неотъемлемую часть жизни пользователей сети Интернет и продолжает увеличивается. По данным аналитического сервиса \flqq SimilarWeb\frqq на 1 апреля 2020 года социальная сеть \flqq Facebook\frqq является самым посещаемым сайтом в категории \flqq Social Networks and Online Communities\frqq, занимая 3 место в мире в общем рейтинге веб-сервисов, уступая поисковой системе \flqq Google\frqq и видеохостингу \flqq YouTube\frqq \cite{cite5}.

	Важно отметить активность сети. Только с 2013 по 2019 год было зарегистрировано более 2 миллиардов уникальных пользователей в социальных сетях \cite{cite1}. Количество активных пользователей Facebook насчитывает более 2,6 миллиарда ежемесячно по данным на первый квартал 2020 года \cite{cite2}.

	Не менее важным игроком сетей Web 2.0 являются мессенджеры. В последнее время они активно набирают популярность, так например, Facebook принял решение разделить социальную сеть и мессенджер. Общение внутри Facebook происходит с помощью специального мессенджера \flqq Messenger\frqq, который занимает второе место по миру в категории \flqq мессенджеры\frqq. Первое место занял \flqq WhatsApp\frqq.

	Мессенджеры к настоящему времени уже занимают первое место по среднему дневному охвату на мобильных устройствах, социальные сети на втором месте. При этом по количеству времени, проведённому внутри приложения, социальные сети лидируют со средним показателем 52 минуты на пользователя в день \cite{cite6}.

	Многие современные мессенджеры ставят целью создание целой экосистемы благодаря реализации базового функционала. Так например, мессенджер \flqq WeChat\frqq, лидер Китайского рынка позволяет
	Лидером Российского рынка является Telegram


Растущим трендом являюстя мессенджеры.  Лидером рынка на данный момент является Facebook,  Активно растущим направлением являются мессенджеры. Так прирост пользователей Telegram за год составил 100 миллионов пользователей, с общим числом пользователей в 400 миллионов (по данным на 24 апреля 2020 года \cite{cite3}). Было оставлено 15 миллиардов сообщений в публичных каналах в период с февраля 2016 года по март 2020 года.

По данным на январь 2019 года, самыми распространёнными мессенджерами стали \cite{cite4}:
1. WhatsApp
2. Facebook Messenger
3. Viber
4. IMO
5. Line
6. Telegram
7. WeChat
8. Google Messenger
9. Hangouts
10. KakaoTalk

Это сети, основной принцип которых является наполнение сети пользовательским контентом, или UGC. Это основное отличие от традиционных сетей. [1: https://www.sciencedirect.com/topics/psychology/user-generated-content]. Таким контентом может являться текстовые сообщения, статьи и посты, комментарии, фотографии, аудио / видео контент и т.д.
На данный момент это самый расрпотсраннёный вид сетей и продолжающий набирать популярность / расти.
Веб 2.0 занимает 90\% рынка, что составляет 7 милилародов пользвоателей по всему миру. Самым популярной сетью является Facebook, это 5 миллиардов пользователей, 3 активных ежедневно, ежедневно пользователи создают 6 мил постов, оставляют 405 милиардов комментариев, публикуют 25 милионов фотографий. На втором месте: Твиттер, далее по рейтингу: Вотсап и на десятом месте Телеграмм.

Ежегодный прирост составляет 500 миллионов пользователей, в то время как прирост Веб 1.0 – 10 миллионов, Веб 3.0 – 2 миллиона.
Весь этот контент содержит какую-то информацию, отображает взаимодействие пользователей – какую-то дискуссию. Дискуссии появляются вокруг какой-то темы, выражаются различные мнения, формируются стороны дискуссии, каждый выражает своё мнение, причём в максимально свободной форме, дискуссии пересекаются или перетекают в другую дискуссию. Всё это представляет особый интерес для анализа. Например дискуссия вокруг какого-то продукта может лучше понять компании потребности и боли своих клиентов. Визуализация по географическим критериям, острота вопроса и объём обсуждения помогут лучше скорректирвоать стратегию компании. То же самое можно сказать и о событиях, новостях, мнениях, происшествиях.
На данный момент нет решений, которые бы проводили комплексный анализ дискуссий, в частности в социальных сетях. Существуют следующие решения: 1.2.3. Это является платным, это было выпущено в 10 году, функционал предполагает построение гарфиков, при этом не поддерживается ряд новых сетей, таких как …
Задача анализа дискуссий в сетях достаточно важна, так как является междисциплинарной, решаются как социально-экономические задачи: таргетинг, экономическое положение дел, политические: анализ реакций населения,  текущих и появляющихся проблем, менеджерских: анализ актуальности идеи и болей пользователей, … Данной задачей занимается специальная наука – Социал нетворк анализис.

Под данной задачей решается множество проблем: выявление проблематики, выделение лидеров мнений, поиск скрытых сообществ, скрытые мотивы, основные объекты проблематики, возможные причины и последствия, актуальность и острота проблемы, отношение пользователей к дискуссии, географическая карта мнений, связанные проблематики.
Не менее важной группой сетей сейчас являются месенджеры. Даже крупные социальные сети отделяют этот функционал в отдельные приложения, как это например сделал Facebook. Множество корпоративных каналов ведётся в Telegram, также эта сеть интересна большим количеством рускоговорящей аудитории и широким функционалом для взаимодействия, аудитория сети – лояльная и имеет высокую конверсию. Форматы взаимодействия сети – разнообразные и позволяют создавать новые форматы взаимодействия с помощью реализации базовых функций. Также многие компании используют эту сеть в виду доступного ПО для внедрения ботов, разворачивании сети сообщества. В сети на данный момент насчитывается миллион каналов с аудиторией больше тысячи пользователей. Администрация сети постоянно ведёт политику по удалению неактивных пользователей и выявление нелегальной деятельности. В данной сети достаточно точно можно провести анализ явлений сети Веб 2.0 на живой аудитории, разнообразном асортименте групп по тематикам и отсутствии мешающей / вносящий погрешность функционал, не поддающийся анализу. Большая часть контента – это сообщения пользователей. Возможно получение информации по гео данным, обращения к другим участникам дискуссии, при этом вся текстовая информация является открытой и доступной для анализа.
В этой работе будет расмотрена возможность выгрузки данного контента и проведение анализа дискуссий пользователей вокруг заданных тематик.




актуальность - экскурс, вводим читателя в состояние дел, занимаюсь этим, явный представить веб 2.0 - сети, там где общаются пользователи, статистика, фейсбук телеграмм, вообще такой задачей вообще есть наука направление - сочиал нетворк анализис, важно так как чвляется задача междисциплинарной, есть социально-эконлмически задачи, цель тартерованный пользователей, поиск скрытых сообществ, хот ворд экстракшн, ..., как пользователи относятся к дискуссии, что обсуждают по разным тематикам, есть ряд проблем - платные / бесплатные / частично, нсть инструменты но не приментильно телеграмма, интересно вытаскивать скрытые данные, проблема с получением данным - анализ дискуссий, применительно месаенджера. Выделяю проблему. Что такое снэй. Какие есть задачи. Свожусь к тому что есть проблемы которые я зочу решать. Вкратце какие попытки у других

Рынок
боль - для репрезентативности, данных слишком много
Запрос на продукт

Ни один сайт не может похвастаться такой активностью как месенджеры (веб 2.0) такой динамикой пользовтаелей, вовлеченностей

Социальная значимость
Пользователи хотят продаигатсья, дополнительный заработок, будут спонсоры

Почему это делают?
Естественная пирамида
Вызывает эмоции
Мы решаем эти глобальные проблемы обозначенные в актуальности (комплексное решение)




В современном мире особо актуален вопрос поиска узких мест и проблем пользователей. Такая информация может определить вектор развития компаний, а также приведёт к возникновению новых рынков и ниш для бизнеса. Во многих современных стартапах используется подход с поиском «болей» пользователей, выдвижением гипотез и их тестированием. Здесь очень важно понимать, что является основополагающим источником проблем и каковы его последствия. Касается это не только бизнеса, но и играет огромную роль в политике и ряде других сфер.  Не менее важным является оценивание реакции на определённые продукты, решения, события. Знание географических и временных рамок очагов проблем могут сыграть ключевую роль в становлении будущего бизнеса и государств. Понимание подхода к решению проблем может качественно изменить оценку решений, вплоть до определения будущих тенденций и предотвращения проблем в корне.

\specialsection{Цель работы}

Анализировать необработанных массив пользовательского контента для выявления зависимости, объектов дискуссии, лидеров мнений, скрытых сообществ, тональности.



- [ ] глобальная цель - создание разработка методов и инструментов для анализа пользовтельских дискуссий в счтех веб 2.0 на примере телеграмм и выявлкние скрытой семантически значимой информации о предмете дискуссий и влиятельных пользователей,
    - [ ] выявление кибербулинг, террористических сообществ, незаконной деятельности, мошеничество
    - [ ] Более крпные организации, политические групировки
    - [ ] Хот ворд эксракшн - заголовки к тексту
    - [ ] Анализ соруктуру дискуссии с выделением горячий объектов (можно использовать н граммы, построить граф и извлечь сущности)
    - [ ] Можно попробовать на примере Бренда



Цель проекта – выявить проблемные темы обсуждения, их интенсивность, а также определить какие тематики являлись источником и следствием.
Задача состоит в том, чтобы собрать необходимое количество данных для анализа из естественного общения людей. Сбор ведётся в рамках определённой области исследования, чтобы получить объект для анализа – посты пользователей в живом обсуждении, выражение эмоций и реакций на публикации других участников диалога. Из собранного массива данных выделить исключительно негативные публикации, поскольку это показывает реальные проблемы и волнующие места для конкретных пользователей. Далее выделить тематики проблемных обсуждений, на основе которых построить тепловую карту конкретной области исследования.

Наиболее эффективным методом для получения реальных мнений людей является сбор данных в сети Интернет. Так как в Интернете меньше всего чувствуется давление и влияние со стороны. Стоит отметить, что данный метод является наиболее экономичным и быстрым способом получения необходимого объёма информации.
Одновременно со сбором данных выгружаются метаданные о пользователях, сообщениях и каналах публикации, что позволит строить географические очаги обсуждений в определённых временных рамках, выделять наиболее популярные источники и ведущих пользователей, а также определять перекрёстных пользователей в обсуждениях и чатах.
Существует множество форумов, мессенджеров и социальных сетей для общения. В каждом сервисе люди по-своему выражают свои чувства и реакции на определённые события. Для анализа в этой работе был выбран – мессенджер Telegram. Этот сервис наиболее интересен своим функционалом и сообществом, которое активно развивает его. В Telegram присутствует большое разнообразие «сущностей», подходов к общению и возможностей для взаимодействия пользователей друг с другом.

\specialsection{Постановка задачи}

Анализ существующих решений
Спроектировать инфраструктуру проекта
Реализовать парсер для выгрузки пользовательского контента
Предварительная обработка данных и подготовка к анализу
Формирование тематик
Визуализация
Тестирование и апробация на реальном кейсе



   - [ ] Для достижения поставленной цели в работе были поставлены следующий задачи:
    - [ ] Провести обзор существующих инструментов по анализу
    - [ ] Алгоритмов по извлечению сущеостей тематики
    - [ ] Спроектировать прзитектуру програмного комплекса для телеграмма на предмет выявления ..
    - [ ] Реализация програмного комплекса
    - [ ] Разработк аметодов анализа данных пользовтелей
    - [ ] Тестирование и опробация на оеальнос кейсе



Проект состоит из нескольких модулей полного цикла – от поиска информации до визуализации аналитических выводов обработанного массива.

2.1 Поиск релевантных источников
Для анализа важно естественное общение пользователей, поэтому далее будут рассматриваться чаты. Мы исключаем каналы, где публикация происходит от одного лица и нет возможности обсуждения.
В Telegram имеется возможность прикреплять чаты к каналам, что позволяет обсуждать опубликованную новость в специальном месте. Помимо этого, возможно прикреплять веб-страницу для комментирования конкретных записей, но данный формат не является популярным.
Для выгрузки сообщений отбираются чаты с относительно большим количеством пользователей (более 1000 человек) и достаточно активным обсуждением – чтобы последние 100 сообщений были сделаны за последнюю неделю. Эти начальные данные для исследования позже могут корректироваться.

2.2 Извлечение данных
Выгрузка публикаций происходит после задания области исследования. Каждая область исследования состоит из ключевых слов, которые определяют эксперты. Например, для области исследования «криптовалюта» - ключевыми словами будут являться «токен», «блокчейн», «распределённый реестр» и т.д.
По каждому из ключевых слов происходит поиск средствами Telegram в отобранных каналах. Стандартный поиск Telegram позволяет находить поисковые запросы в разных формах и склонениях.

2.3 Обработка данных
На этапе обработки данных происходит приведение естественных текстов в более формальный, структурированный вид – удобный для анализа.
Обработка состоит из следующих этапов:
1.	Лемматизация – все слова приводятся в начальную форму.
2.	Производится поиск шинглов – биграмм и триграмм, т.е. часто повторяющихся словосочетаний в текстах.
3.	Каждая публикация разбивается на список слов (шинглов) и создаётся «мешок слов», по которому далее будет происходить индексирование.
4.	Отбираются стоп-слова – слова не имеющие отношения к объекту анализа.
5.	Фильтрация по части речи (числительные, союзы, предлоги и т.д. исключаются).
6.	Фильтрация по частотности – отсекаются слишком редкие и слишком частые слова (шинглы).
7.	По итогу преобразований – отсеиваются слишком короткие, непоказательные, сообщения.
8.	С помощью сентимент-анализа определяется тональность текста, далее отсеиваются публикации с положительной и нейтральной тональностью.

2.4 Формирование тематик
Для задачи выявления тематик используется алгоритм LDA (Latent Dirichlet allocation). Тема формируется из списка слов и коэффициентов вхождения этого слова в тематику. Стоит отметить, что любой документ (публикация) с какой-то долей вероятностью соотносится с каждой из выявленных тематик. В данной работе для простоты анализа принято считать, что документ относится к наиболее вероятной тематике.

2.5 Визуализация
Проблемные темы раскладываются по временной шкале и определяется их актуальность в заданный промежуток времени. Результаты анализа обобщаются в таблице, именуемой «Тепловая карта». Каждая ячейка подсвечивается цветом – от холодного к тёплому. Благодаря такой визуализации понятно какие темы были актуальны, как долго и в какие следующие темы переходил интерес.

\specialsection{Практическая значимость}

- [ ] Статистика о вовлеченности прльзовтеедй в месаенджеры
    - [ ] Модео сфокусироваться на бренд менеджменте - дополнительный источник

\section{Обзор существующих решений}
\subsection{Технологический обзор}

    - [ ] Технологический обзор (какие готовые сервисы уже есть которые занимаются анадизом , данными торгуют твиттер фейсбук, на каком уровне глубины проводят анализ)
    - [ ] Мы рассмотрим потенциал глубокого анализа

\subsection{Обзор существующих методов анализа}

    - [ ] Общор существующих методов анализа дискусстй
        - [ ] Можно выявлять скрытые сообщества
        - [ ] Инфлюентеров
        - [ ] Эмоции детектировать
        - [ ] Разные лда лса
        - [ ] Академический обзор

\section{Разработка программного комплекса}
\subsection{Проектирование архитектуры}

Абстракция
Для начала необходимо определить какие форматы взаимодействия подразумевают нынешние UGC-системы
Посты - пост, сообщение, ветка обсуждения, статья
Реакции - лайки, комментарии, дизлайки, смайлики, стикеры, кастомные реакции, просмотры, репосты
Каналы - паблики, группы, каналы, авторы, пользователи
Чаты - чаты, супергруппы, обсуждения
Профили - имя, фамилия, описание, геопозиция, логин, юзер-нейм, ник, личная информация, связь с другими пользователями, подписки, подписчики, оставленные посты, ссылки на аккаунты в других сетях



    - [ ] Проектирование арзитектуры
    - [ ] Описание каждого модуля
    - [ ] Стек технологий
    - [ ] Прокктирование и реализация

\subsection{Извлечение пользовательского контента}

   - [ ] Непосоедственно анализ
    - [ ] Методы анализа дискуссий пользователя в телеграмм
    - [ ] Собственные методы
    - [ ] Какую информацию собираю
    - [ ] Какую статистку считаю



Мною разработано веб-приложение, в рамках которого происходит получение контента, анализ и визуализация. Данное решение предпринято ввиду удобства доступа и взаимодействия с приложением из браузера на любом интернет устройстве в адаптивном формате, а также ввиду ограниченного доступа к ресурсам, необходимым для анализа.

3.1 Тепловая карта
Поскольку обработка каждого запроса происходит продолжительное время, то интерфейс был разбит на пункты «Добавить на обработку» и «Обработанные запросы» (рис. 1).


Рисунок 1. Интерфейс запросов

Можно добавить новый запрос на обработку, после чего он отобразится в списке обработанных запросов. Время обработки зависит от сложности запроса, в среднем это занимает 15 минут.
Публикации и метаданные выгружаются в объектно-ориентированную базу данных MongoDB. С этой целью созданы соответствующие коллекции «requests» – для каждого запроса (здесь хранится информация о ключевых словах, статус обработки, результаты обработки – доступная временная шкала, список тематик и актуальность тематик на временной шкале) и «messages» – для выгрузки сообщений из Telegram и быстрого доступа (хранится тело сообщения, источник, идентификаторы сообщения, автор, время публикации, принадлежность к запросу и результаты обработки – список слов в начальной форме, если это сообщение было отобрано для дальнейшего анализа).

При выборе запроса в списке обработанных, открывается страница с тепловой картой (рис. 2).


Рисунок 2. Тепловая карта

В первом столбце перечислены выявленные тематики, состоящие из слов и коэффициента его вхождения. Далее по временной шкале расписаны в процентном соотношении актуальность темы в текущий промежуток времени – от синего до бордового. Каждая ячейка – это один месяц временной шкалы.

В нижеприведённом примере запроса (рис. 3) виден переход интереса обсуждения из одной тематики в другую.


Рисунок 3. Тепловая карта с выделением перехода в другую тему

3.2 График активности
Также была реализована возможность получить график активности обсуждения (рис. 4).


Рисунок 4. График активности

\subsection{Подготовка данных для анализа}

\subsection{Анализ}

\subsection{Визуализация}

\subsection{Опрабация}

    - [ ] Тестирование и оценка качества
    - [ ] Или опробация
Подпункт 1.
    - [ ] Постановка эксперимента
    - [ ] Берем кейс по
    - [ ] Были такие штуки, требовалось найти каналы с каждого канала собрать такую мнформацтю и провести такой анализ
    - [ ] Построить словарь, битерм, топинг модулинг, бренды
Подпункт 2.
    - [ ] Результаты эксперимента
    - [ ] Числа (значения метрик алгоритмов)
    - [ ] Аналитике данных
    - [ ] В этой дискуссий св получили вот такой частотный словарь
    - [ ] Построили граф понятий хот вордов горячих тем
    - [ ] Покзаать доказательство сводки - на таймсериенс вот здесь стало актуально, вот здесь пики, столько пощитивных, частота упоминаемости
    - [ ] Отеосительн рбрендов построить динамику присутсвия в постах
    - [ ] Какие основные зотворды крутятся вокруг этих брендов
    - [ ] Говорить компаниям вот о вас такое обсуждают

Подпункт 3.
    - [ ] Выводы
    - [ ] Способен решать такие задачи
    - [ ] Что наши данные могут показать
    - [ ] Это говнит о такой динамике
    - [ ] Это может послуживать информацией для марктеологов / экспертов
    - [ ] Самые обсуждаемые эти
    - [ ] Самые менее обсуждаемые



Полученный результат по сообщениям Telegram совпадает с графиком аналитического сервиса «Google Trends» по количеству запросов в поисковой строке (рис. 5).


Рисунок 5. Google Trends

\section{Заключение}
\subsection{Результаты работы}

      - [ ] В прошедшем времени задачи которые ставили - спроектирована архитектура, реализована, подобран лучший алгоритм для детектирования скрытых сообществ
        - [ ] Проведен обзор существующих решений / алгоритмов и направлений данноой предметной области
        - [ ] Выполнено тестирование т опрабация на реальных данных
        - [ ] Которые показал такие результаты

\subsection{Перспективы развития}

        - [ ] Что могли бы улучшить
        - [ ] Для полноты сбора



В перспективе планируется выявить и реализовать наиболее точные методы аналитики. В частности, реализовать и изучить результаты BTM (Biterm Topic Modeling), который предназначен для анализа на малых фрагментах данных, подобно текстовым сообщениям в мессенджерах.
	В целях исключения таких явлений, как реклама, «вбросы» и сообщения, не являющиеся форматом естественного общения, необходимо реализовать алгоритм фильтрации контента. Также необходимо отсеивать повторяющиеся сообщения.
	Реализовать граф взаимосвязи слов, чтобы выделить сущности и основные объекты в коллекции.
	Сохранять больше информации о сообщениях, добавить коллекции для каналов и пользователей в базе данных. Это позволит учитывать перекрёстных пользователей, а также географическую причастность.
	Реализовать более гибкий анализ с регулировкой параметров количества сообщений, выборки каналов и визуализации: по временной шкале и количеству тематик. Также необходимо будет ввести возможность указывать несколько ключевых слов и словосочетаний в рамках одной области исследования.




1.	Bodrunova S. S., Blekanov I. S., Kukarkin M. M. «Topics in the Russian Twitter and Relations between their Interpretability and Sentiment» https://ieeexplore.ieee.org/abstract/document/8931725 // 2019
2.	Topic modeling visualization – How to present the results of LDA models? https://www.machinelearningplus.com/nlp/topic-modeling-visualization-how-to-present-results-lda-models/ // 2018
3.	Bakharia A. «Topic Modeling with Scikit Learn» https://medium.com/mlreview/topic-modeling-with-scikit-learn-e80d33668730 // 01.09.2016
4.	Садреева Ю. И., Добрынин В. Ю. Выпускная квалификационная работа бакалавра «Автоматическая классификация новостей из коллекции Reuters в таксономию IPTC» https://dspace.spbu.ru/bitstream/11701/4113/1/VKR\_Sadreeva.pdf // 2016
5.	LDA in Python – How to grid search best topic models? https://www.machinelearningplus.com/nlp/topic-modeling-python-sklearn-examples/ // 2017



Ненумерованная формула:

\begin{equation}
    \begin{pmatrix} \dot{\varphi}\\ \dot{\theta} \\ \dot{\psi} \end{pmatrix}
    = \begin{pmatrix}
        cos(\theta)cos(\psi) & -sin(\psi) & 0 \\
        cos(\theta)sin(\psi) & cos(\psi)  & 0 \\
        -sin(\theta)         & 0         &  1
    \end{pmatrix}^{-1}
    \begin{pmatrix} \omega_x\\ \omega_y \\ \omega_z \end{pmatrix}.
\end{equation}

Нумерованная формула:

\begin{equation}
    i^2 = -1.
    \label{eq:my_ref}
\end{equation}

Тест ссылки на формулу \ref{eq:my_ref}.

Ниже тестируется очень большая таблица на несколько страниц

\begin{center}
    \begin{longtable}{|p{2cm}|p{3cm}|p{7cm}|p{3cm}|}
    \caption{Заголовок таблицы}\\
    \hline
    1 & 2 & 3 & 4\\
    \hline
    2 & 2 & 3 & 4\\
    \hline
    3 & 2 & 3 & 4\\
    \hline
    4 & 2 & 3 & 4\\
    \hline
    5 & 2 & 3 & 4\\
    \hline
    6 & 2 & 3 & 4\\
    \hline
    7 & 2 & 3 & 4\\
    \hline
    8 & 2 & 3 & 4\\
    \hline
    9 & 2 & 3 & 4\\
    \hline
    10 & 2 & 3 & 4\\
    \hline


    \end{longtable}
\end{center}


А также тестируется счетчик таблиц, жирные и двойные линии.

\begin{center}
    \begin{longtable}{|p{2cm}||p{3cm}|p{7cm}|p{3cm}|}
    \caption{Заголовок таблицы нумер 2}\\
    \hline
    1 & 2 & 3 & 4\\
    \hline
    2 & 2 & 3 & 4\\
    \hline
    3 & 2 & очень жирная ячейка \par с переносом (работаеттт!) & 4\\
    \hline
    4 & 2 & 3 & 4\\
    \hline
    5 & 2 & 3 & 4\\
    \hline
    6 & 2 & 3 & 4\\
    \hline
    7 & 2 & 3 & 4\\
    \hline
    8 & 2 & 3 & 4\\
    \hline
    9 & 2 & 3 & 4\\
    \hline
    10 & 2 & 3 & 4\\
    \hline


    \end{longtable}
\end{center}

\pagebreak

\begin{figure}[ht]
\begin{center}
\scalebox{0.4}{
   \includegraphics{images/graph.jpg}
}

\caption{
\label{graph-fig}
     Линейные функции.}
\end {center}
\end {figure}
Ссылаемся на график ~\ref{graph-fig}.

% Библиография в cpsconf стиле
% Аргумент {1} ниже включает переопределенный стиль с выравниванием слева
\begin{thebibliography}{1}
\bibitem{cite1} Lopez‐Castroman J, Moulahi B, Azé J, et al. \flqq Mining social networks to improve suicide prevention: A scoping review\frqq. J Neurosci Res. 2020; 98:616–625.

\bibitem{cite2} Facebook Reports First Quarter 2020 Results // Facebook. [2019] URL: \href{https://s21.q4cdn.com/399680738/files/doc\_financials/2020/q1/Q1-2020-FB-Earnings-Presentation.pdf}{https://s21.q4cdn.com/399680738/files/doc\_financials/2020/q1/Q1-2020-FB-Earnings-Presentation.pdf} (дата обращения: 27.05.2020)

\bibitem{cite3} Telegram Reports // Telegram. [2020] Дата обновления: 24.04.2020. URL: \href{https://telegram.org/blog/400-million?ln=f}{https://telegram.org/blog/400-million?ln=f} (дата обращения 28.05.2020)

\bibitem{cite4} Global Digital Report 2019 // We Are Social. [2008-2020] URL: \href{https://wearesocial.com/global-digital-report-2019}{https://wearesocial.com/global-digital-report-2019} (дата обращения: 28.05.2020)

\bibitem{cite5} Top Websites Ranking // SimilarWeb [2020]. Дата обновления: 01.04.2020. URL: \href{https://www.similarweb.com/top-websites}{https://www.similarweb.com/top-websites} (дата обращения: 28.05.2020)

\bibitem{cite6} Екатерина Курносова, Социальные сети в цифрах // Российский интернет-форум | РИФ+КИБ, 2019. URL: \href{https://mediascope.net/upload/iblock/f97/18.04.2019\_Mediascope\_\%D0\%95\%D0\%BA\%D0\%B0\%D1\%82\%D0\%B5\%D1\%80\%D0\%B8\%D0\%BD\%D0\%B0\%20\%D0\%9A\%D1\%83\%D1\%80\%D0\%BD\%D0\%BE\%D1\%81\%D0\%BE\%D0\%B2\%D0\%B0\_\%D0\%A0\%D0\%98\%D0\%A4+\%D0\%9A\%D0\%98\%D0\%91\%202019.pdf}{https://mediascope.net/upload/iblock/f97/18.04.2019\_Mediascope\_Екатерина Курносова\_РИФ+КИБ 2019.pdf} (дата обращения 28.05.2020)

\end{thebibliography}
\end{document}