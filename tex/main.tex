\documentclass{article}

% Свои операторы
\usepackage{amsmath}
\DeclareMathOperator{\tg}{tg}

%
\usepackage{natbib}
\usepackage{graphicx}

% Кодировка
\usepackage[utf8]{inputenc}

% Для написаний
\usepackage{amssymb}

% Русский язык
\usepackage[russian]{babel}

% Шрифт
\renewcommand{\familydefault}{\sfdefault}

% Графики
% \usepackage{pgfplots}
% \RequirePackage{luatex85,shellesc}

% Ссылки
\usepackage{hyperref}


\title{Проект}
\author{Полоз А. Е.}
\date{2019}


\begin{document}
	
\maketitle

\section{Введение}
\label{section:introduction}
\subsection{Понятие}
\begin{itemize}
	\item Пункт 1
	\item  Пункт 2
\end{itemize}

\noindent\rule{\textwidth}{1pt}

$x \mapsto \in \mathbb{R} \frac{1}{x-1} \backslash \{1\} \Delta x_0 f^\prime (1 + x)^\frac{1}{x} \infty \tg \alpha \rightarrow$ 
$$\lim_{x \to a} f(x) =$$

\hyperref[section:introduction]{ссылка}\\

\begin{tabular}{ | l | r | }
	\hline
	Хедер 1 & Хедер 2 \\ \hline
	строка 1, столбец 1 & строка 1, столбец 2 \\
	\hline
\end{tabular}

\newpage
\bibliographystyle{plain}
\bibliography{references}
\end{document}